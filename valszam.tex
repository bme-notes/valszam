%%%% Beállítások, importok

\input{includes/header.tex}

\usepackage{relsize}
\usepackage{pdfpages}

\newcommand{\Gorog}[1]{\mathlarger{\mathlarger{\mathlarger{#1}}}}
\newcommand{\Vegtelen}{\limits_{-\infty}^{+\infty}}
\newcommand{\EX}{\textbf{E}X}
\newcommand{\EY}{\textbf{E}Y}
\newcommand{\szoras}{\Gorog{\sigma}X}
\newcommand{\Szoras}{\Gorog{\sigma}^2X}
\newcommand{\forceindent}{\leavevmode{\parindent=1em\indent}}

%%%%
%%%%%%%
%%%% Ezeket változtasd meg!

\cim{Valószínűségszámítás}
\datum{2017. január 9.}
\szerzo{Kormány Zsolt}
\segitettek{Bognár Márton}

%%%%
%%%%%%%
%%%% Fedlap

\begin{document}
\begin{titlepage}
		\centering
		\vspace{5cm}\par
		\maketitle
		\large A jegyzet és annak forrása megtalálható a \texttt{bme-notes.github.io} weboldalon.
		\vfill
		Közreműködtek: \the\segitettek
		\normalsize
		% Bottom of the page
\end{titlepage}

%%%%
%%%%%%%
%%%% Tartalomjegyzés + előszó

\noindent \textbf{Kellemes vizsgázást!}

\tableofcontents{}

\section{Előszó}
\href{http://vik.wiki/images/e/e7/Valszam_konyv_olvashato_kela.pdf}{Szamozas Kela 1998-as konyve szerint}

%%%%
%%%%%%%
%%%% Tényleges content
%%%% A fejezeteket a fejezetek/$NEV.tex elérési útvonalon keresi
%% Első "fejezet"
\ujfejezet{valoszinusegi-mezo}
%% Második "fejezet"
\ujfejezet{valoszinusegi-valtozok}
\ujfejezet{valoszinusegi-vektorvaltozok}
\ujfejezet{nagyszamok}
\ujfejezet{tulelo}

\end{document}
