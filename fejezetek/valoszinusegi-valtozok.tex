\section{A Valózínűségi váltózók}
\subsection{Általánosan}
\begin{tetel}{MZ/X használhatatlan cuccokat küld}
II.2.2 A valószinűségi változó eloszlása és eloszlásfüggvénye kölcsönösen egyértelműen meghatározzák egymást. \\[8pt]
\end{tetel}

\begin{tetel}{MZ/X használhatatlan cuccokat küld}
	II.2.3 ($F_x(x)$ eloszlásfüggvény tulajdonságai)
	\begin{itemize}
		\item $F_x$ monoton nemcsökkenő, azaz $F_x(x) \leq F_x(y), ha\, x < y.$
		\item $F_x$ balról folytonos
		\item $\lim\limits_{x \to +\infty} F_x(x) = 1$ és $\lim\limits_{x \to -\infty} F_x(x) = 0$
	\end{itemize}
\end{tetel}

	\begin{definicio}{Mézga Géza}
   ($f_X(x)$ sűrűségfüggvény tulajdonságai
	\begin{itemize}
		\item $f_X(t) \geq 0$ \qquad ( $\Leftarrow F_x \uparrow$ )
		\item $\int\Vegtelen f_X(t) dt = 1 $ \qquad ( $ \Leftarrow \lim\limits_{t  \to \infty} F_X(t) = 1 )$
	\end{itemize}
\end{definicio}

	%TODO ricsi táblázatát az eloszlásokról beszúrni, illetve a normális eloszlásról 1-2 dolgot írni
	\textbf{Örökifjúak}: Geometriai , Exponenciális

\subsection{A Várható érték}
	\begin{definicio}{Mézga Géza}
  II.6.1 \textbf{Várható érték létezése} (Diszkrét,\small folytonos ugyanez csak integral) \\[3pt]\normalsize
		$\sum\limits_{i=1}^\infty |x_i| \cdot P(X = x_i) Konvergens \Longrightarrow \exists EX$
		$$ \textbf{E}X = \sum_{i=1}^\infty x_i \cdot P(X = x_i)$$

\end{definicio}
	%TODO Műveleti tulajdonságok!!!

\subsection{Magasabb momentumok, szórásnégyzet}

  \begin{definicio}{Mézga Géza}
	II.7.1 \textbf{n-edik momentum} \\[3pt]
	Az X valószínűségi változó n-edik momentumán az $X^n$ valószínűségi változó várható értékét értjük, ha létezik. \\[3pt]
	$$Jel: \mu_n =  \textbf{E}X^n = \int\limits_{-\infty}^{+\infty} x^n \cdot f_X(x) dx$$\\[3pt]
  \end{definicio}
  \begin{definicio}{Mézga Géza}
	II.7.2 \textbf{Szórásnégyzet}\\[3pt]
		Az X valószínűségi változó szórásnégyzetén vagy varianciáján az $ Y = (X - EX)^2$ valószínűségi változó várható értékét értjük( amennyiben létezik)
	$$\Gorog{\sigma}^2X = \textbf{E}\Big( (X-\textbf{E}X)^2 \Big) $$\\[3pt]
  \end{definicio}

  \begin{tetel}{MZ/X használhatatlan cuccokat küld}
	II.7.2 \textbf{Steiner formula}\\[-6pt]

	$$\Gorog{\sigma}^2X = \textbf{E}X^2 - (\textbf{E}X)^2 $$\\[3pt]

  \end{tetel}

	%TODO MUveleti tulajdonságok a végére
	\begin{definicio}{Mézga Géza}
  II.7.3 \textbf{standardizált}\\[3pt]
	Az X valószínűségi változó standardizáltján az $\widetilde{X} = \dfrac{X-\EX}{\szoras} $ lineárisan transzformált valószínűségi változót értjük
\end{definicio}
